% !Tex TS-program = xelatex
% !TEX encoding = UTF-8 Unicode

\documentclass[
12pt, a4paper,% tamanho da fonte e papel.
openright,% capítulos começam em pág ímpar (insere página vazia caso preciso)
oneside,% para impressão em recto somente. Oposto a twoside.
english,
brazil	% o último idioma é o principal do documento, ademais são hifenizados corretamente.
]{abntex2}
\RequireXeTeX %Force XeTeX check

% --- (tudo que vem depois de '%' é um comentário em latex)
% ---
% Pacotes fundamentais 
% ---
\usepackage{lmodern}% Usa a fonte Latin Modern
\usepackage[T1]{fontenc}% Seleção de codigos de fonte.
\usepackage[utf8]{inputenc}% Codificacao do documento (acentos)
\usepackage{indentfirst}% Indenta o primeiro parágrafo da secção.
\usepackage{color}% Controle das cores
\usepackage{graphicx}% Inclusão de gráficos
\usepackage{microtype}% para melhorias de
% justificação
\usepackage{xltxtra} %fontspec, metalogo e realscripts (XeLaTex)
\usepackage{fontspec}
\usepackage{lipsum} % Enche linguíça (preenche espaço)
\usepackage[alf]{abntex2cite}% Citações padrão ABNT
\usepackage{amsmath} % Diversas tipografias matemáticas

\begin{document} %% Iniciar o documento

\chapter{Capítulo 1}
\section{Secção número 1.1}

\textbf{De acordo com \cite{foo}, Literate programming
  é o paradigma mais formal e divertido de todos - ao vivo.}

\begin{figure}[ht]
  \centering
  \caption{\label{fig:lt1} Exemplo de literate programming.}
  \includegraphics[width=\linewidth]{../img/literate-programming.jpg}
  \legend{Reference: The internet}
\end{figure}

\lipsum[1-2] % Texto enche linguíça

\bibliography{arquivo-com-bibliografias} % Usar bibliografias

\end{document}
