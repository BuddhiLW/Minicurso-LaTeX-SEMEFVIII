% Created 2021-09-25 Sat 20:00
% Intended LaTeX compiler: pdflatex
\documentclass[11pt]{article}
\usepackage[utf8]{inputenc}
\usepackage[T1]{fontenc}
\usepackage{graphicx}
\usepackage{grffile}
\usepackage{longtable}
\usepackage{wrapfig}
\usepackage{rotating}
\usepackage[normalem]{ulem}
\usepackage{amsmath}
\usepackage{textcomp}
\usepackage{amssymb}
\usepackage{capt-of}
\usepackage{hyperref}
\usepackage[hidelinks]{hyperref}
\usepackage{minted}
\usemintedstyle{lovelace}
\date{\today}
\title{}
\hypersetup{
 pdfauthor={},
 pdftitle={},
 pdfkeywords={},
 pdfsubject={},
 pdfcreator={Emacs 27.2 (Org mode 9.4.4)}, 
 pdflang={English}}
\begin{document}

\tableofcontents


\section{Comandos utilizados}
\label{sec:orgd6fb0e9}
\subsection{Classe do documento (comando)}
\label{sec:org84fe096}
\begin{minted}[fontsize=\scriptsize,autogobble=true,frame=lines,framesep=1mm,linenos=false]{latex}
\documentclass[bigger]{beamer}
\end{minted}

\subsection{Importação de pacotes (comando)}
\label{sec:orgde9c935}

Para qualquer pacote listado no \url{https://www.ctan.org/}, basta chamarmos
o seguinte comando para incorporá-lo no documento em mãos.

\begin{minted}[fontsize=\scriptsize,autogobble=true,frame=lines,framesep=1mm,linenos=false]{latex}
\usepackage{<nome-do-pacote>}
\end{minted}

\subsubsection{Estilizações de um pacote}
\label{sec:orgd86dd73}

\begin{enumerate}
\item Exemplo \texttt{tcolorbox}
\label{sec:org5251a90}
Alguns pacotes possuem estilização interna ao pacote. Por exemplo,
\texttt{tcolorbox}, um pacote utilizado para estilizar blocos e tabelas, pode
requerir a chamada de um "subpacote" adicional.

Exemplo:
Na documentação, é dito, para importação de bibliotecas extras, dentro
de tcolorbox,
\url{http://mirrors.ctan.org/macros/latex/contrib/tcolorbox/tcolorbox.pdf}
É dito, utilize essa sintaxe.

\begin{minted}[fontsize=\scriptsize,autogobble=true,frame=lines,framesep=1mm,linenos=false]{latex}
\tcbuselibrary{⟨key list⟩}
\end{minted}

Todas as opções:

\href{img/libraries-tcolobox.png}{\includegraphics[width=0.5\textwidth]{/home/buddhilw/PP/LaTeX/SEMEF-minicurso/Apres1/img/libraries-tcolobox.png}}

\item Caso de uso, na apresentação:
\label{sec:orgde3448e}

No \texttt{preâmbulo}, chamei o pacote e biblioteca
\begin{minted}[fontsize=\scriptsize,autogobble=true,frame=lines,framesep=1mm,linenos=false]{latex}
\usepackage{tcolorbox}
\tcbuselibrary{skins}
\end{minted}

Adaptei um exemplo da documentação,
\begin{minted}[fontsize=\scriptsize,autogobble=true,frame=lines,framesep=1mm,linenos=false]{latex}
\newenvironment{modern-quote}{\begin{itemize}}{\end{itemize}}
\tcolorboxenvironment{modern-quote}{blanker,
  before skip=6pt,
  after skip=6pt,
  borderline west={3mm}{0pt}{red}}
\end{minted}

Por fim, para chamar o novo ambiente no documento, temos,
\begin{minted}[fontsize=\scriptsize,autogobble=true,frame=lines,framesep=1mm,linenos=false]{latex}
\begin{modern-quote}
"Don't give up on your dreams, keep on sleeping."
\end{modern-quote}
\end{minted}

\begin{enumerate}
\item Renderização:
\label{sec:org505dcc1}

\href{img/modern-quote.png}{\includegraphics[width=0.4\textwidth]{/home/buddhilw/PP/LaTeX/SEMEF-minicurso/Apres1/img/modern-quote.png}}
\end{enumerate}
\end{enumerate}

\subsubsection{Todos pacotes utilizados na apresentação, com suas opções e bibliotecas.}
\label{sec:org24d8811}
\begin{minted}[fontsize=\scriptsize,autogobble=true,frame=lines,framesep=1mm,linenos=false]{latex}
\usepackage[utf8]{inputenc}
\usepackage[T1]{fontenc}
\usepackage{graphicx}
\usepackage{grffile}
\usepackage{longtable}
\usepackage{wrapfig}
\usepackage{rotating}
\usepackage[normalem]{ulem}
\usepackage{amsmath}
\usepackage{textcomp}
\usepackage{amssymb}
\usepackage{capt-of}
\usepackage{hyperref}
\author[Branquinho]{Pedro Gomes Branquinho \\ \text{\scriptsize{pedro.branquinho@usp.br}}}
\date[EEL-USP]{\scriptsize{Mini-curso de \LaTeX} \\ Universidade de São Paulo - DEMAR}
\usepackage{pifont}
\usepackage{verbatim}
\makeatletter
\def\verbatim@font{\scriptsize\ttfamily}
\makeatother
\logo{\includegraphics[height=0.5cm]{./img/usp-logo-1}}
\AtBeginSubsection[]{\begin{frame}\frametitle{Table of Contents}
    \tableofcontents[currentsection,currentsubsection]
  \end{frame}}
\usepackage{tikz}
\usetikzlibrary{arrows.meta}
\usetikzlibrary{positioning}
\usepackage{tcolorbox}
\tcbuselibrary{skins}
\usepackage{minted}
\usemintedstyle{lovelace}
\newenvironment{modern-quote}{\begin{itemize}}{\end{itemize}}
\tcolorboxenvironment{modern-quote}{blanker,before skip=6pt,after skip=6pt, borderline west={3mm}{0pt}{red}}
{\usebackgroundtemplate{\includegraphics[width=\paperwidth]{./img/TP-opacity-50.jpg}}
  \usetheme[height=20pt]{Hannover}
  \usecolortheme{seahorse}
  \date{  Universidade de São Paulo - DEMAR}
  \title{O \LaTeX{} e suas funcionalidades}
  \setbeamertemplate{itemize item}{\ding{166}}
  \setbeamercolor{item projected}{bg=magenta!90!black,fg=white}
  \setbeamertemplate{enumerate item}[circle]
  \setbeamercolor{block title}{bg=red!30!white,fg=black}
  \hypersetup{
    pdfauthor={},
    pdftitle={O \LaTeX{} e suas funcionalidades},
    pdfkeywords={},
    pdfsubject={},
    pdfcreator={Emacs 27.2 (Org mode 9.4.4)}, 
    pdflang={Portuguese}}
\end{minted}

\subsection{Inicialização de slides (ambiente)}
\label{sec:org7146984}
\subsubsection{Uso}
\label{sec:org334bae9}
\begin{minted}[fontsize=\scriptsize,autogobble=true,frame=lines,framesep=1mm,linenos=false]{latex}
\begin{frame}[label={sec:org03cdf3a},fragile]{Origem de \TeX{} - Knuth (1978)}
  (...)
\end{frame}
\end{minted}

\subsubsection{Renderização}
\label{sec:orgf341f99}
\href{img/frame.png}{\includegraphics[width=0.3\textwidth]{/home/buddhilw/PP/LaTeX/SEMEF-minicurso/Apres1/img/frame.png}}

\subsection{Compartição de slides em colunas (ambientes)}
\label{sec:orgba84341}
\subsubsection{Uso}
\label{sec:org4fcb77a}

Dentro do ambiente \texttt{frame}, teremos um ambiente \texttt{columns}, o qual pode
ser dividido em diversas colunas unitárias, com  \texttt{column}. Nesse caso,
cada uma ocupou 48\% do espaço disponível para a coluna

\begin{minted}[fontsize=\scriptsize,autogobble=true,frame=lines,framesep=1mm,linenos=false]{latex}
\begin{frame}[label={sec:org03cdf3a},fragile]{Origem de \TeX{} - Knuth (1978)}
  %%%%%% Ambiente interior a um frame
  \begin{columns}

    %%% * Coluna 1
    \begin{column}{0.48\columnwidth}
      (...) % Imagem
    \end{column}

    %%% * Coluna 2
    \begin{column}{0.48\columnwidth} 
      (...) % Código tipografado
    \end{column}

  \end{columns}
  %%%%%% 
\end{frame}
\end{minted}
\subsubsection{Renderização}
\label{sec:orgd6090ba}
\href{img/colunas.png}{\includegraphics[width=0.5\textwidth]{/home/buddhilw/PP/LaTeX/SEMEF-minicurso/Apres1/img/colunas.png}}

\subsection{Blocos de texto (ambiente)}
\label{sec:org77f3b4d}
No preâmbulo, especificação de fonte preta e fundo rosa claro.
\begin{minted}[fontsize=\scriptsize,autogobble=true,frame=lines,framesep=1mm,linenos=false]{latex}
\setbeamercolor{block title}{bg=red!30!white,fg=black}
\end{minted}

\subsubsection{Exemplo e renderização}
\label{sec:org4ce1153}
No corpo do documento
\begin{minted}[fontsize=\scriptsize,autogobble=true,frame=lines,framesep=1mm,linenos=false]{latex}
\begin{block}{\small{~Ille eruditus et sapiens~}}
(...)
\end{block}
\end{minted}

\href{img/titulo-bloco.png}{\includegraphics[width=0.3\textwidth]{/home/buddhilw/PP/LaTeX/SEMEF-minicurso/Apres1/img/titulo-bloco.png}}
\subsection{Itemização (ambiente)}
\label{sec:orge4e1c2f}
\subsubsection{Customização da itemização}
\label{sec:org06450fe}
No preâmbulo, itens de símbolo ding 166.
\begin{minted}[fontsize=\scriptsize,autogobble=true,frame=lines,framesep=1mm,linenos=false]{latex}
\usepackage{pifont} %Alguns símbolos unicode (dingbats)
\setbeamertemplate{itemize item}{\ding{166}} % Todo itemize será renderizado com esse símbolo 166.
\end{minted}
\begin{enumerate}
\item Uso e renderização
\label{sec:org1b9c90d}
Nota: as opções \texttt{[<+->]}, fazem com que os itens apareçam
gradualmente, a cada slide consecutivo.

\begin{minted}[fontsize=\scriptsize,autogobble=true,frame=lines,framesep=1mm,linenos=false]{latex}
\begin{itemize}[<+->]
\item Preview em tempo real.
(...)
\end{itemize}
\end{minted}

\href{img/ding.png}{\includegraphics[width=0.4\textwidth]{/home/buddhilw/PP/LaTeX/SEMEF-minicurso/Apres1/img/ding.png}}
\end{enumerate}

\subsubsection{Customização do enumerate}
\label{sec:org5eeeb37}
No preâmbulo, elementos redondos e rosas.
\begin{minted}[fontsize=\scriptsize,autogobble=true,frame=lines,framesep=1mm,linenos=false]{latex}
\setbeamercolor{item projected}{bg=magenta!90!black,fg=white} % elementos enumerados estilizados (magenta escura)
\setbeamertemplate{enumerate item}[circle] %enumeração com fundo redondo
\end{minted}

\begin{enumerate}
\item Uso e renderização
\label{sec:orga05a210}
\begin{minted}[fontsize=\scriptsize,autogobble=true,frame=lines,framesep=1mm,linenos=false]{latex}
\begin{enumerate}
\item Primeiro item
\item Segundo item
\end{enumerate}
\end{minted}

\href{img/enumerate.png}{\includegraphics[width=0.2\textwidth]{/home/buddhilw/PP/LaTeX/SEMEF-minicurso/Apres1/img/enumerate.png}}
\end{enumerate}
\subsection{Imagens}
\label{sec:orgcf13cd0}
\subsubsection{Uso}
\label{sec:org8685d88}

Dentro de uma \texttt{coluna}, em um \texttt{frame}, temos um \texttt{bloco} com uma
\texttt{imagem}.

\begin{minted}[fontsize=\scriptsize,autogobble=true,frame=lines,framesep=1mm,linenos=false]{latex}
\begin{block}<1->{Imagem Lamport}
  \includegraphics[width=1.02\textwidth]{./img/Leslie_Lamport.jpg}
\end{block}
\end{minted}

\subsubsection{Renderização}
\label{sec:org38a7f12}
\href{img/bloco-image.png}{\includegraphics[width=0.3\textwidth]{/home/buddhilw/PP/LaTeX/SEMEF-minicurso/Apres1/img/bloco-image.png}}
\subsection{Equações}
\label{sec:orgb22f5c0}
\subsubsection{Uso}
\label{sec:orgb4df39a}
Em ambientes \texttt{equation}, tipografamos uma equação,
\begin{minted}[fontsize=\scriptsize,autogobble=true,frame=lines,framesep=1mm,linenos=false]{latex}
\begin{equation}
  \begin{aligned}
    \dfrac{\partial{\vec{V}}}{\partial{t}}
    + \vec{V}.\nabla{\vec{V}}
    = - \dfrac{\nabla{p}}{\rho}
    + \nu{}\nabla^2{\vec{V}}
  \end{aligned}
\end{equation}
\end{minted}

\subsubsection{Renderização}
\label{sec:org2ef9fe4}

\begin{enumerate}
\item Por imagem
\label{sec:orgd5ef9b9}

\href{img/navier.png}{\includegraphics[width=\textwidth]{/home/buddhilw/PP/LaTeX/SEMEF-minicurso/Apres1/img/navier.png}}

\item Por código
\label{sec:orgbe83253}

\begin{equation}
  \begin{aligned}
    \dfrac{\partial{\vec{V}}}{\partial{t}}
    + \vec{V}.\nabla{\vec{V}}
    = - \dfrac{\nabla{p}}{\rho}
    + \nu{}\nabla^2{\vec{V}}
  \end{aligned}
\end{equation}
\end{enumerate}
\subsection{Tabelas (ambiente)}
\label{sec:org3c37492}
\subsubsection{Uso}
\label{sec:orgebb6e31}
Notas:
\begin{itemize}
\item O ambiente pode ser \texttt{tabular} or \texttt{table}
\item \texttt{\textbackslash{}(x=y\textbackslash{})} é equivalente a \texttt{\$x=y\$}. Ou seja, \texttt{\$} e \texttt{\textbackslash{}(} ou \texttt{\textbackslash{})} são intercambiáveis. Porém, não deve os utilizar parcialmente. Isso é, o código \texttt{\$x=y\textbackslash{})} não renderiza corretamente.
\item \texttt{\{lll\}} significa, três elementos por linha, cada um alinhado à esquerda (left).
\item \texttt{\&} são separadores de elementos, por coluna.
\item \texttt{\textbackslash{}\textbackslash{}} quebram linhas
\item \texttt{\textbackslash{}hline} significa uma linha horizontal ``(h[orizontal]line)''
\end{itemize}

\begin{minted}[fontsize=\scriptsize,autogobble=true,frame=lines,framesep=1mm,linenos=false]{latex}
\begin{tabular}{lll}
  \hline
  Coluna 1 & Coluna 2 & Coluna 3\\
  \hline
  \(a_{11}\) & \(a_{12}\) & \(a_{13}\)\\
  \(a_{21}\) & \(a_{22}\) & \(a_{23}\)\\
  Texto 1 & Texto 2 & Texto 3\\
  \hline
\end{tabular}
\end{minted}

Outro exemplo, mas de acordo com as normas abnt

\begin{minted}[fontsize=\scriptsize,autogobble=true,frame=lines,framesep=1mm,linenos=false]{latex}
\begin{table}[htb]
  \begin{center}

    \ABNTEXfontereduzida

    \caption[<como aparece na lista de tabelas>]{\label{tab:formal} Formatação Tipográfica, modelo de
      tabela genérica}

    \begin{tabular}{m{2.6cm}|m{4.0cm}|m{2.25cm}|m{3.40cm}}
      % \hline
      \textbf{Pretendemos} & \textbf{Temos} & \textbf{Em \LaTeX{}es} & \textbf{Alternativamente}\\
      \hline
      Serif & {\rmfamily\textbf{R}o\textbf{m}ana} & \verb+{\rmfamily}+  & \verb+\textrm{}+ \\
      \hline
      Sans Serif & {\sffamily{\textbf{S}ans Serif\textbf{f}} & \verb+{\sffamily}+  & \verb+\textsf{}+\\
      \hline
      Type Writer & {\ttfamily{\textbf{T}ype Wri\textbf{t}er}}  & \verb+{\ttfamily}+ & \verb+\texttt{}+\\
      \hline
    \end{tabular}
    \legend{Fonte: o autor}

  \end{center}
\end{table}
\end{minted}


\subsubsection{Renderiza}
\label{sec:orgc5130eb}
\begin{enumerate}
\item Por imagem
\label{sec:orge70ae10}

\href{img/tabela-img.png}{\includegraphics[width=0.5\textwidth]{/home/buddhilw/PP/LaTeX/SEMEF-minicurso/Apres1/img/tabela-img.png}}

\item Por código
\label{sec:org8254ad7}
\begin{enumerate}
\item Exemplo 1
\label{sec:org08d47fb}

\begin{tabular}{lll}
  \hline
  Coluna 1 & Coluna 2 & Coluna 3\\
  \hline
  \(a_{11}\) & \(a_{12}\) & \(a_{13}\)\\
  \(a_{21}\) & \(a_{22}\) & \(a_{23}\)\\
  Texto 1 & Texto 2 & Texto 3\\
  \hline
\end{tabular}
\item Exemplo 2
\label{sec:org9eb1b0d}

\begin{table}[htb]
  \begin{center}

    \ABNTEXfontereduzida

    \caption[<como aparece na lista de tabelas>]{\label{tab:formal} Formatação Tipográfica, modelo de
      tabela genérica}

    \begin{tabular}{|m{2.6cm}|m{4.0cm}|m{2.25cm}|m{3.40cm}}
      \hline
      \textbf{Pretendemos} & \textbf{Temos} & \textbf{Em \LaTeX{}es} & \textbf{Alternativamente}\\
      \hline
      Serif & {\rmfamily\textbf{R}o\textbf{m}ana} & \verb+{\rmfamily}+  & \verb+\textrm{}+ \\
      \hline
      Sans Serif & {\sffamily{\textbf{S}ans Serif\textbf{f}} & \verb+{\sffamily}+  & \verb+\textsf{}+\\
      \hline
      Type Writer & {\ttfamily{\textbf{T}ype Wri\textbf{t}er}}  & \verb+{\ttfamily}+ & \verb+\texttt{}+\\
      \hline
    \end{tabular}
    \legend{Fonte: o autor}

  \end{center}
\end{table}
\end{enumerate}
\end{enumerate}

\section{Vocábulos e notações}
\label{sec:org2a8e11e}

\begin{center}
\begin{tabular}{ll}
\hline
Abreviação & Significado\\
\hline
\texttt{bg} & background.\\
\texttt{fg} & foreground.\\
\hline
\end{tabular}
\end{center}
\end{document}