% Created 2021-09-26 Sun 16:22
% Intended LaTeX compiler: pdflatex
\documentclass[11pt]{article}
\usepackage[utf8]{inputenc}
\usepackage[T1]{fontenc}
\usepackage{graphicx}
\usepackage{grffile}
\usepackage{longtable}
\usepackage{wrapfig}
\usepackage{rotating}
\usepackage[normalem]{ulem}
\usepackage{amsmath}
\usepackage{textcomp}
\usepackage{amssymb}
\usepackage{capt-of}
\usepackage[draft]{hyperref}
\usepackage{minted}
\usemintedstyle{lovelace}
\date{\today}
\title{}
\hypersetup{
 pdfauthor={},
 pdftitle={},
 pdfkeywords={},
 pdfsubject={},
 pdfcreator={Emacs 27.2 (Org mode 9.4.4)}, 
 pdflang={English}}
\begin{document}

\tableofcontents


\section{Distribuições existentes (com vasto suporte)}
\label{sec:org5361ea1}
Essas são algumas das distribuições, além da proposta pelo autor, que podem ser adotadas.
\subsection{Spacemacs}
\label{sec:org2474e8e}
\begin{itemize}
\item \url{https://www.spacemacs.org/}
\end{itemize}
\subsection{DOOM Emacs}
\label{sec:org0bee1b7}
\begin{itemize}
\item \url{https://github.com/hlissner/doom-emacs}
\end{itemize}
\subsection{Prelude Emacs}
\label{sec:orgb51cd62}
\begin{itemize}
\item \url{https://github.com/bbatsov/prelude}
\end{itemize}

\section{Como instalar a versão utilizada na SEMEF?}
\label{sec:org9b300e2}
\subsection{Git}
\label{sec:org93fb830}
\subsubsection{Instale o Git}
\label{sec:orgddc796b}
\url{https://git-scm.com/downloads}
\subsubsection{Clone o repositório SEMEF (Frank-init)}
\label{sec:org7a03ef5}
Abra um \texttt{terminal Git}, na pasta \texttt{C:\textbackslash{}<Usuários>\textbackslash{}<Seu-Usuário>}. Isso
pode ser feito entrando na pasta e clicando com o botão direito do
mouse, e selecionando a opção do \texttt{terminal Git}.

Escreva o comando:

\begin{minted}[fontsize=\scriptsize,autogobble=true,frame=lines,framesep=1mm,linenos=false]{shell}
git clone https://github.com/BuddhiLW/Frank-init/tree/SEMEF .emacs.d
\end{minted}

Se ja tiver uma pasta chamada \texttt{.emacs.d} no diretório, a delete, ou a
renomeia para outro nome, e.g., \texttt{.emacs.d.backup}. Em seguida, dê o
comando acima.

\subsubsection{Clone o \texttt{ob-julia}}
\label{sec:orgbc51835}
\begin{enumerate}
\item Clonando \texttt{ob-julia}
\label{sec:org7854ad0}
Navegue ao diretório .emacs.d, pelo terminal,
\begin{minted}[fontsize=\scriptsize,autogobble=true,frame=lines,framesep=1mm,linenos=false]{shell}
cd .emacs.d
\end{minted}

obs: \texttt{cd} := \texttt{change directory}.

Dentro de \texttt{.emacs.d},
\begin{minted}[fontsize=\scriptsize,autogobble=true,frame=lines,framesep=1mm,linenos=false]{shell}
git clone https://github.com/shg/ob-julia-vterm.el
\end{minted}
\end{enumerate}

\subsubsection{Assistir como utilizar o Emacs}
\label{sec:orgd7f1b7a}

Teoricamente, você deveria ter assistido o tutorial de como instalar o
Emacs, com esse vídeo (onde alguns comandos do Emacs são comentados):
\begin{itemize}
\item \url{https://www.youtube.com/watch?v=TEqhxzfYXKA}
\end{itemize}

Agora, é hora de assistir como utilizar o Emacs, propriamente:
\begin{itemize}
\item \url{https://www.youtube.com/watch?v=ub3T-y6\_u-4}
\end{itemize}

Recomendação, após o vídeo dedicado ao emacs:

\begin{itemize}
\item \textbf{O que é e como utilizar o básico de org-mode}
\begin{itemize}
\item \url{https://www.youtube.com/watch?v=ffBDQauDAgQ}
\end{itemize}
\end{itemize}
\end{document}